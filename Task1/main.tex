\documentclass{article}
\usepackage{graphicx} % Required for inserting images
\usepackage{amsmath}
\usepackage{geometry} % Required for adjusting margins
\usepackage[T2A]{fontenc}
\usepackage[utf8]{inputenc}
\usepackage[russian]{babel}
\usepackage[russian]{babel}

% Set custom margins
\geometry{left=2cm, right=2cm, top=2cm, bottom=2cm}

\title{домашнее задание 1}
\author{Чилеше Абрахам}
\date{April 2024}

\begin{document}
	\maketitle

	\newpage

	% QUESTION ONE-------------------------------------------
	\raggedright
	\section{Найти все значения корня $\sqrt[4]{1}$ : }

	\textbf{Главное значение (основное значение)}:
	\[
		(1)^{\frac{1}{4}}= e^{\frac{1}{4} \cdot \ln(1)}= e^{0}= 1
	\]

	\textbf{Тригонометрическая форма}:
	\[
		1 = e^{i0}
	\]
	При нахождении четвертого корня из 1 в комплексном анализе мы делим аргумент на
	4 и добавляем $\frac{2\pi k}{4}$, где $k = 0, 1, 2, 3$, чтобы получить все четыре
	корня.

	\begin{itemize}
		\item Первое значение: $e^{i\frac{0}{4}}= e^{i0}= 1$

		\item Второе значение: $e^{i\frac{2\pi}{4}}= e^{i\frac{\pi}{2}}= i$

		\item Третье значение: $e^{i\frac{4\pi}{4}}= e^{i\pi}= -1$

		\item Четвертое значение: $e^{i\frac{6\pi}{4}}= e^{i\frac{3\pi}{2}}= -i$
	\end{itemize}

	\[
		\boxed{\text{Ответ: }\sqrt[4]{1} \ = $1$, $i$, $-1$ и $-i$}
	\]

	\vspace{1cm} % Add space here

	% QUESTION ONE-------------------------------------------
	\section{Представить в алгебраической форме \operatorname{ch} \left(2 +
	\frac{\pi i}{2}\right): }
	Исходное выражение:
	\[
		\text{ch}(2 + \frac{\pi i}{2})
	\]

	Используем определение гиперболического косинуса (ch) через экспоненту:
	\[
		\text{ch}(z) = \frac{e^{z} + e^{-z}}{2}
	\]

	Подставим $z = 2 + \frac{\pi i}{2}$ в определение:
	\[
		\text{ch}(2 + \frac{\pi i}{2}) = \frac{e^{2 + \frac{\pi i}{2}}+ e^{-(2 +
		\frac{\pi i}{2})}}{2}
	\]

	Распишем экспоненты:
	\[
		e^{2 + \frac{\pi i}{2}}= e^{2} \cdot e^{\frac{\pi i}{2}}
	\]
	\[
		e^{-(2 + \frac{\pi i}{2})}= e^{-2}\cdot e^{-\frac{\pi i}{2}}
	\]

	Подставим выражения для экспонент в исходное выражение:
	\[
		\text{ch}(2 + \frac{\pi i}{2}) = \frac{e^{2} \cdot e^{\frac{\pi i}{2}}+ e^{-2}\cdot
		e^{-\frac{\pi i}{2}}}{2}
	\]

	Воспользуемся формулой Эйлера для экспоненты:
	\[
		e^{i\theta}= \cos(\theta) + i\sin(\theta)
	\]

	Получаем:
	\[
		e^{\frac{\pi i}{2}}= i
	\]
	\[
		e^{-\frac{\pi i}{2}}= -i
	\]

	Подставим значения экспонент обратно в исходное выражение:
	\[
		\text{ch}(2 + \frac{\pi i}{2}) = \frac{e^{2} \cdot i + e^{-2}\cdot (-i)}{2}
	\]

	Упростим:
	\[
		\text{ch}(2 + \frac{\pi i}{2}) = \frac{e^{2} \cdot i - e^{-2}\cdot i}{2}
	\]

	Финальный ответ:
	\[
		\boxed{\text{ch}(2 + \frac{\pi i}{2}) = \frac{(e^{2} - e^{-2}) \cdot i}{2}}
	\]

	\vspace{1cm}

	%QUESTION 3--------------------------------------------------------

	\section{Представить в алгебраической форме: arcsin(4)}
	\vspace{0.5cm}
	\[
		\arcsin(z) = -i \Ln(iz + \sqrt{1 - z^{2}})
	\]
	\[
		\arcsin(4) = -i \Ln(4i + \sqrt{1 - 4^{2}}) =
	\]
	\[
		-i \Ln(4i + \sqrt{-15}) = -i \Ln(4i + \sqrt{-15}) = \boxed{ -i \Ln((4 + \sqrt{15}) \cdot i) }
	\]

	\vspace{0.5cm}

	%QUESTION 4----------------------------------------------------
	\section{Представить в алгебраической форме: \text{(-1 + i)^{-3i}}}
	\[
		(-1 + i)^{-3i}= e^{-3i\ln(-1 + i)}
	\]

	\[
		\ln(-1 + i) = \ln(|-1 + i|) + i \text{Arg}(-1 + i)
	\]
	\[
		= \ln\sqrt{2}+ i \text{Arg}(-1 + i) = \ln\sqrt{2}+ i \left( \frac{3\pi}{4}+ 2
		\pi n \right), \quad n \in \mathbb{Z}
	\]

	Таким образом, получаем: \\
	\[
		(-1 + i)^{-3i}= e^{-3i(\ln\sqrt{2} + i ( \frac{3\pi}{4} + 2\pi n) }= e^{3(\frac{3\pi}{4}
		+ 2\pi n)}(\cos(3\ln{\sqrt{2}}) + i\sin(3\ln{\sqrt{2}})
	\]

	\[
		\boxed{e^{3(\frac{3\pi}{4} + 2\pi n)} (\cos(3\ln{\sqrt{2}}) + i\sin(3\ln{\sqrt{2}})}
	\]

	\newpage

	% QUESTION 5----------------------------------------------------
	\section{Представить в алгебраической форме: $\text{Ln}(2+i)$}

	\[
		\text{Ln}(2+i) = \ln|2 + i| + i\text{Arg}(2+i) = \ln 5 + i\left(\text{arg}(2+
		i) + 2\pi k\right), k \in \mathbb{Z}
	\]

	\[
		\boxed{\text{Ответ:}\quad\ln 5 + i\left(\arctan(\frac{1}{2}) + 2\pi k\right), k \in \mathbb{Z}}
	\]

	\vspace{0.5cm}
	% QUESTION 6----------------------------------------------------
	\section{Вычертить область, заданную неравенствами:
	\newline
	D = {\text{Z: |Z - i|} \leq 2, Rez \geq 1}}

	\begin{enumerate}
		\[
			z: | z - i | = 2
		\]
		Здесь мы начинаем с выражения в комплексной форме: $z$.
		\newline
		Модуль $z - i$ устанавливается равным 2.

		\[
			z: | x + yi - i | = 2
		\]
		Мы подставляем $z = x + yi$ в выражение.
		\newline
		$z - i$ представляет собой разницу между $z$ и $i$ (т.е. расстояние от $i$ на
		комплексной плоскости).

		\[
			z: | x + (y - 1)i | = 2
		\]
		Упрощаем $yi - i$ для получения $(y - 1)i$.
		\newline
		Теперь $x$ представляет собой действительную часть $z$, а $(y - 1)$
		представляет собой мнимую часть $z$ относительно положения $i$ вдоль мнимой
		оси.

		\[
			z : \sqrt{x^{2} + (y - 1)^{2}}= 2
		\]
		Мы выражаем модуль как квадратный корень из суммы квадратов действительной и
		мнимой частей.
		\newline
		Это формула расстояния на комплексной плоскости.

		\[
			z : x^{2} + (y - 1)^{2} = 2^{2}
		\]
		Возводим обе части уравнения в квадрат, чтобы убрать знак квадратного корня.
		\newline
		Этот шаг упрощает уравнение.

		\[
			z : x^{2} + (y - 1)^{2} = 4
		\]
		Упрощаем $2^{2}$ для получения $4$.
		\newline
		\newline
		\vspace{0.5cm}
		\centering
		\includegraphics[width=0.3\textwidth]{graph.jpg}
		\centering
	\end{enumerate}
	\vspace{0.7cm}
	\newpage
	\section{ Определить вид пути и в случае, когда он проходит через точку 8, исследовать
	его поведение в этой точке:
	\newline
	z = -4sh5t - i5ch5t}

	\section{Определить тип особой точки $z = 0$ для данной функции: $f(z) = z^{4}\cos
	(\frac{5}{z^{2}})$}

	% QUESTION 12 %
	\section{12. Найти все лорановское разложение данной функции по степеням
	$z - z_{0}$: $f(z) = \frac{z - 1}{z(z + 1)},\  z_{0} = -1 + 2i$}

	Преобразуем Данную функцию; \\
	\vspace{0.2cm}
	\[
		f(z) = \frac{z-1}{z(z+1)}= \frac{2}{z+1}- \frac{1}{z}
	\]\\
	\vspace{0.4cm}
	Используем разложения в ряд Тейлора в окрестностм точки $z_{0}: \\
	\vspace{0.2cm}$$\frac{1}{z+a}= \frac{1}{a}- \frac{z}{a^{2}}+ \frac{z^{2}}{a^{3}}
	-\frac{z^{3}}{a^{4}}+ ... = \sum_{n=0}^{\infty}\frac{(-1)^{n}z^{n}}{a^{n+1}}$$ $$\frac{2}{z+1}
	= 2 \cdot \frac{1}{z+1}= 2 \cdot \frac{1}{(z-z_{0}) + 2i}= 2 \cdot \sum_{n=0}^{\infty}
	\frac{(-1)^{n}(z-z_{0})^{n}}{(2i)^{n+1}}$$ $$\frac{1}{z}= \frac{1}{(z-z_{0}) -
	1 +2i}= \sum_{n=0}^{\infty}\frac{(-1)^{n}(z-z_{0})^{n}}{(2i - 1)^{n+1}}$$Таким
	образом:$$f(z) = \frac{2}{z-1}- \frac{1}{z}= 2 \cdot \sum_{n=0}^{\infty}\frac{(-1)^{n}(z-z_{0})^{n}}{(2i)^{n+1}}
	- \sum_{n=0}^{\infty}\frac{(-1)^{n}(z-z_{0})^{n}}{(2i - 1)^{n+1}}= (-1)^{n}\sum
	_{n=0}^{\infty}\left[\frac{2}{(2i)^{n+1}}- \frac{1}{(2i - 1)^{n+1}}] \right](z-
	z_{0})^{n}$$Ответ:$$(-1)^{n}\sum_{n=0}^{\infty}\left[\frac{2}{(2i)^{n+1}}- \frac{1}{(2i
	- 1)^{n+1}}] \right](z-z_{0})^{n}$$\vspace{0.8cm}% QUESTION 13 %
	\section{13. Данную функцию разложить в ряд Лорана в окрестности точки $z_{0}$:
	$\quad f(z)=z\sin{\frac{z^{2} -2z}{(z-1)^{2}}}, z_{0} = 1$}
	\\ Перейдем к новой переменной$z' = z-z_0$ $$z' = z-1; z\sin{\frac{z^{2} -2z}{(z-1)^{2}}}
	= (z'+1)sin\frac{z'^{2} + 1}{z'^{2}}= (z'+1)sin\left( 1 + \frac{1}{z'^{ 2}}\right
	) =$$ $$z'\sin{1}\cos{\frac{1}{z'^{2}}}+ z'\cos{1}\sin{\frac{1}{z'^{2}}}+ sin1c
	os\frac{1}{z'^{2}}+ \cos{1}\sin{\frac{1}{z'^{2}}}= f(z')$$Теперь нам остается н
	айти разложение получившейся функции от$z'$в окрестности точки$z'_0=0$. Для это
	го следует использовать табличные разложения в ряд Тейлора:$$f('z) = z'\sin{1}\cos
	{\frac{1}{z'^{2}}}+ z'\cos{1}\sin{\frac{1}{z'^{2}}}+ \sin{1}\cos{\frac{1}{z'^{2}}}
	+ \cos{1}\sin{\frac{1}{z'^{2}}}=$$ $$= \left( 1 - \frac{1}{2!z'^{4}}+ \frac{1}{4!z'^{8}}
	- \frac{1}{6!z'^{12}}+ ... \right)z'\sin{1}+ \left( \frac{1}{z'^{2}}- \frac{1}{3!z'^{6}}
	- \frac{1}{5!z'^{10}}- ...\right)z'\cos{1}$$ $$\left( 1 - \frac{1}{2!z'^{4}}+ \frac{1}{4!z'^{8}}
	- \frac{1}{6!z'^{12}}+ ... \right)\sin{1}+ \left( \frac{1}{z'^{2}}- \frac{1}{3!z'^{6}}
	- \frac{1}{5!z'^{10}}- ...\right)\cos{1}=$$ $$z'sin1 + sin1 + \frac{cos1}{z'}+
	\frac{cos1}{z'^{2}}- \frac{sin1}{2!z'^{3}}- \frac{sin1}{2!z'^{4}}- \frac{cos1}{3!z'^{5}}
	- \frac{cos1}{3!z'^{6}}+ ...
	\vspace{0.4cm}$$Произведем обратную замену переменной и, таким образом, получим
	разложение исходной функции в ряд Лорана в окрестности точки$z_0=1$ $$f(z) = z\sin
	{1}+ \frac{cos1}{z-1}+ \frac{cos1}{(z-1)^{2}}- \frac{sin1}{2!(z-1)^{3}}- \frac{sin1}{2!(z-1)^{4}}
	- \frac{cos1}{3!(z-1)^{5}}- \frac{cos1}{3!(z-1)^{6}}+ ...$$Ответ:$$f(z) = z\sin
	{1}+ \frac{cos1}{z-1}+ \frac{cos1}{(z-1)^{2}}- \frac{sin1}{2!(z-1)^{3}}- \frac{sin1}{2!(z-1)^{4}}
	- \frac{cos1}{3!(z-1)^{5}}- \frac{cos1}{3!(z-1)^{6}}+ ...$$% QUESTION 15 %
	\section{15. Для данной функции найти изолированные особые точки и определить
	их тип $f(z) = ctg\frac{1}{z}$}
	\text{Перейдем к новой переменной}\\$t = \frac{1}{z}; f(t) = ctgt
	\vspace{0.4cm}
	$\text{Эта функция не является аналитической при sin t = 0. Найдем t, соответствующие
	этому случаю }\\$t = \frac{1}{z}; f(t) = ctgt
	\vspace{0.4cm}
	$ $\sin{t} = 0 \Longrightarrow t = \pi k; k\in z $\text{Запишем данную функцию
	в виде отношения функций g(t) и h(t):}\\$ f(t) = \frac{\cos{t}}{\sin{t}}; g(t)
	= cost; h(t) = sint
	\vspace{0.4cm}
	$Для каждой из функций найдем порядок производной, не обращающейся в ноль при$t
	= \pi k$:
	\vspace{0.4cm}
	\\$ g(\pi k) \neq 0 \\ h(\pi k) = 0 \\ h'(t) = \cos{t}; h'(\pi k) \neq 0
	\vspace{0.4cm}
	$Так как порядок производной, не обращающейся в ноль при$t = \pi k$выше для фун
	кции, находящейся в знаменателе, то точки$t = \pi k$являются полюсами функции.
	Порядок этих полюсов находится, как разница между порядками производных, не обр
	ащающихся в ноль при$t = \pi k$для функций$h(t)$и$g(t)$. В данном случае, это$1
	- 0 = 1$\vspace{0.2cm}$ t = \pi k \longrightarrow z = \frac{1}{t} =
	\frac{1}{\pi k}; k \in z$\vspace{0.4cm}
	Рассмотрим точку$z = 0$. Для любого$\delta>0$существует такое значение k, что$
	| 1/\pi k | < \delta$. Таким образом$z = 0$не является изолированной особой точ
	кой, так как противоречит определению, гласящему, что функция должна быть анали
	тической в некотором кольце вокруг этой точки, а, какой бы мы не взяли радиус к
	ольца, в нем найдется особая точка вида$1/\pi k$, в которой функция не является
	аналитической \\
	\vspace{0.4cm}
	Ответ: Точки$z=1/\pi k;k \in z $ для данной функции являются полюсами 1-го
	порядка. 11
\end{document}
