\documentclass{article}

\usepackage[T2A]{fontenc}
\usepackage[utf8]{inputenc}
\usepackage[english, russian]{babel}
\usepackage{graphicx}
\usepackage{amsmath}
\usepackage{amssymb}
\usepackage{cancel}
\usepackage{amsfonts}
\usepackage{titlesec}
\usepackage{titling}
\usepackage{geometry}
\usepackage{pgfplots}
\usepackage{esint}
\pgfplotsset{compat=1.9}

\titleformat{\section}
{\normalfont\Large\bfseries}{\arabic{section}}{1em}{}
\titleformat{\subsection}
{\normalfont\large\bfseries}{}{1em}{}

\setlength{\droptitle}{-3em}
\title{\vspace{-1cm}
ИДЗ №2}
\author{Чилеше Абрахам}
\date{Группа: Б9122-02-03-01сцт}

\geometry{a4paper, margin=2cm}

\begin{document}
	\maketitle

	\section{Вычислить интеграл:}
	\[
		\oint_{|z| = 1}\frac{2 + \sin{z}}{z(z + 2i)}\, dz
	\]

	\subsection{Решение:}
	Найдем Особые точки:\\ $z = 0$.\\ $z = -2i$.\\

	\raggedright Точка $z = 2i$ не входит в область, ограниченную данным контуром,
	поэтому не рассматривается.
	\newline
	\raggedright Точка $z_{1} = 0$ является простым полюсом. Найдем вычет в этой
	точке.

	\[
		\underset{z_1}{\text{res}}\ f(z) = \lim\limits_{z\rightarrow 0}\left[f(z) (z
		- 0) \right] = \lim\limits_{z\rightarrow 0}\frac{2z + z\sin{z}}{z(z + 2i)}= \frac{2
		+ \sin{z}}{z + 2i}= \frac{1}{i}= -i
	\]

	\raggedright Отсюда следующий результат.
	\newline
	\[
		\oint_{|z| = 1}\frac{2 + \sin{z}}{z(z + 2i)}\ dz = 2\pi i \sum_{k}^{i=1}\underset
		{z_1}{\text{res}}\ f(z) = 2\pi i.(-i) = 2\pi
	\]

	\subsection{Ответ:}
	\[
		\raggedright{\oint_{|z| = 1} \frac{2 + \sin{z}}{z(z + 2i)}\, dz = 2\pi}
	\]

	\vspace{1.5cm}

	\section{Вычислить интеграл: }
	\[
		\oint_{|z| = 3}\frac{1 - \sin{\frac{1}{z}}}{z}\, dz
	\]

	\subsection{Решение:}

	\section{Вычислить интеграл: }
	\[
		\ointop_{|z| =0,2}\frac{3\pi z - \sin{3\pi z}}{-z^{2}\sh^{2}\pi^{2}z}\ dz
	\]
	\subsection{Решение:}
	\raggedright Особые точки этой функций $z =ik/\pi$. Однако в контур попадает
	только z = 0. Определим тип этой особой точки:
	\newline
	\raggedright
	\[
		f(z) = \frac{3\pi z - \sin{3\pi z}}{-z^{2}\sh^{2}\pi^{2}z}= \frac{g(z)}{h(z)}
	\]
	\[
		g(z) = 3\pi z - \sin{3\pi z}, \quad \quad h(z) = -z^{2}\sh^{2}\pi^{2}z
	\]

	\[
		\lim\limits_{z\rightarrow 0}[f(z)z] = \lim\limits_{z\rightarrow 0}\left(\frac{3\pi
		z - \sin{3\pi z}}{-z\sh^{2}\pi^{2}z}\right) =
	\]

	\[
		\lim\limits_{z\rightarrow 0}\left(\frac{3\pi - 3\pi\cos{3\pi z }}{-\sh^{2}\pi^{2}z
		- 2\pi^{2}z\sh(\pi^{2}z)\ch(\pi^{2}z)}\right) =
	\]

	\[
		\lim\limits_{z\rightarrow 0}\left(\frac{9\pi^{2}\sin{3\pi z}}{-4\pi^{2}\sh\pi^{2}z\ch\pi^{2}z
		- 4\pi^{4}z\ch^{2}\pi^{2}z + 2\pi^{4}z}\right) =
	\]

	\[
		\lim\limits_{z\rightarrow 0}\left(\frac{27\pi^{4}cos{3\pi z}}{-12\pi^{4}\ch^{2}\pi^{2}z
		+ 6\pi^{4}-8\pi^{6}zch\pi^{2}z\sh\pi^{2}z}\right) = -\frac{9}{2\pi}
	\]
	\newline
	По основной теореме Коши о вычетах:

	\[
		\ointop_{|z| =0,2}\frac{3\pi z - \sin{3\pi z}}{-z^{2}\sh^{2}\pi^{2}z}\ dz = 2
		\pi i \sum\limits_{i=1}^{n}resf(z) = 2\pi i. \left(-\frac{9}{2\pi}\right) = -
		9i
	\]

	\subsection{Ответ:
	\[
		\ointop_{|z| =0,2}\frac{3\pi z - \sin{3\pi z}}{-z^{2}\sh^{2}\pi^{2}z}\ dz = -
		9i
	\]
	}

	\vspace{1.5cm}

	\section{Вычислить интеграл: }

	\section{Вычислить интеграл: }
	\[
		\int_{0}^{2\pi}\frac{dt}{3 - 5\sqrt{5}\sin{t}}
	\]

	\subsection{Решение:}
	Преобразуем в контурный интеграл, используя следующие преобразования:
	\[
		z = e^{\pi}; \quad \cos{t}= \frac{1}{2}\left( z + \frac{1}{z}\right); \quad \sin
		{t}= \frac{1}{2i}\left( z - \frac{1}{z}\right); \quad dt = \frac{dz}{iz}
	\]
	\[
		\int_{0}^{2\pi}R(\cos{t}, \sin{t}) ,\ dt = \ointop_{|z| = 1}F(z) ,\ dz
	\]
	\[
		\int_{0}^{2\pi}\frac{dt}{3 - 5\sqrt{5}\sin{t}}= \ointop_{|z| = 1}\frac{\frac{dz}{iz}}{5
		-\frac{\sqrt{5}}{2i}(z - \frac{1}{z})}= \ointop_{|z| = 1}\frac{\frac{dz}{iz}}{3iz
		-\frac{\sqrt{5}}{2}(z^{2}- 1)}= \ointop_{|z| = 1}\frac{2dz}{6iz -\sqrt{5}(z^{2}-
		1)}=
	\]
	\[
		\ointop_{|z| = 1}\frac{2dz}{-\sqrt{5}(z - i/\sqrt{5})(z - i\sqrt{5})}
	\]
	Таким образом, подынтегральная функция имеет 2 особые точки:
	$z = \frac{i}{\sqrt{5}}; \quad z = i\sqrt{5}$ \\ Точка $z = i\sqrt{5}$ не
	попадает в область, ограниченную контуром интегрирования. \\ Точка
	$z = \frac{i}{\sqrt{5}}$ является простым полюсом. Необходимо вычислить вычет в
	этой точке:

	\[
		\underset{z = \frac{i}{\sqrt{5}}}{\text{res}}\ f(z) = \lim\limits_{z\rightarrow
		\frac{i}{\sqrt{5}}}\left[ f(z)(z -\frac{i}{\sqrt{5}})\right] =
	\]
	\[
		\lim\limits_{z\rightarrow \frac{i}{\sqrt{5}}}\frac{2}{-\sqrt{5}(z - i\sqrt{5}}
		) = \frac{2}{-\sqrt{5}(\frac{i}{\sqrt{5}} -i\sqrt{(}5)}= -\frac{i}{2}
	\]
	По основной теореме Коши о вычетах получаем:
	\[
		\ointop_{|z| = 1}\frac{2dz}{-\sqrt{5}(z -\frac{i}{\sqrt{5}}) (z - i\sqrt{5})}
		= 2\pi i \sum\limits_{i = 1}^{n}resf(z) = 2\pi i \cdot (-\frac{i}{2}) = \pi
	\]
	\subsection{Ответ:
	\[
		\int_{0}^{2\pi}\frac{dt}{3 - 5\sqrt{5}\sin{t}}= \pi
	\]
	}

	\section{Вычислить интеграл: }
	\[
		\int_{0}^{2\pi}\frac{dt}{(\sqrt{5}+ \cos{t})^{2}}
	\]
	\subsection{Решение:}
	Преобразуем в контурный интеграл, используя следующие преобразования:
	\[
		z = e^{\pi}; \quad \cos{t}= \frac{1}{2}\left( z + \frac{1}{z}\right); \quad \sin
		{t}= \frac{1}{2i}\left( z - \frac{1}{z}\right); \quad dt = \frac{dz}{iz}
	\]
	\[
		\int_{0}^{2\pi}R(\cos{t}, \sin{t}) ,\ dt = \ointop_{|z| = 1}F(z) ,\ dz
	\]
	\[
		\int_{0}^{2\pi}\frac{dt}{(\sqrt{5}+ \cos{t})^{2}}= \ointop_{|z| = 1}\frac{\frac{dz}{iz}}{(\sqrt{5}+
		\frac{{1}}{2}(z + \frac{1}{z}))^{2}}= \ointop_{|z| = 1}\frac{zdz}{i(z\sqrt{5}+
		\frac{1}{2}(z^{2} + 1))^{2}}= \ointop_{|z| = 1}\frac{4zdz}{i \left[(z + \sqrt{5}+
		2) (z + \sqrt{5}- 2)\right]^{2}}
	\]
	Таким образом, подынтегральная функция имеет 2 особые точки: $z = 2 - \sqrt{5};
	\quad z = -2 - \sqrt{5}$ \\ Точка $z = 2 - \sqrt{5}$ не попадает в область,
	ограниченную контуром интегрирования. \\ Точка $z = -2 - \sqrt{5}$ является полюсом
	второго порядка. Необходимо вычислить вычет в этой точке:
	\[
		\underset{z = 2 - \sqrt{5}}{\text{res}}\ f(z) = \lim\limits_{z\rightarrow 2 -
		\sqrt{5}}\frac{d}{dz}\left[ f(z)(z - 2 + \sqrt{5})^{2}\right] = \lim\limits_{z
		= 2 - \sqrt{5}}\frac{d}{dz}\frac{4z}{i \left[(z + \sqrt{5}+2)^{2} \right] }=
	\]
	\[
		\frac{4}{i}\cdot \lim\limits_{z = 2 - \sqrt{5}}\frac{d}{dz}\frac{z}{ \left[(z
		+ \sqrt{5}+2)^{2} \right] }= \frac{4}{i}\cdot \lim\limits_{z = 2 - \sqrt{5}}\frac{2
		+ \sqrt{5} - z}{ (z + \sqrt{5}+2)^{3} }= \frac{4}{i}\cdot.\frac{2 + \sqrt{5}
		- 2 + \sqrt{5}}{ (2 + \sqrt{5}+2 - \sqrt{5})^{3} }= \frac{4}{i}\cdot\frac{2\sqrt{5}}{4^{3}}
		= \frac{\sqrt{5}}{8i}
	\]
	По основной теореме Коши о вычетах получаем:
	\[
		\ointop_{|z| = 1}\frac{4zdz}{i \left[(z + \sqrt{5}+ 2) (z + \sqrt{5}- 2)\right]^{2}}
		= 2\pi i \sum\limits_{i = 1}^{n}resf(z) = 2\pi i \cdot \frac{\sqrt{5}}{8i}= \frac{\sqrt{5}}{4}
		\pi
	\]
	\subsection{Ответ: $\frac{\sqrt{5}}{4}\pi$}

	\vspace{1cm}

	\section{Вычислить интеграл: }
	\[
		\int_{-\infty}^{+\infty}\frac{dx}{(x^{2} - x + 1)^{2}}
	\]
	\subsection{Решение:}
	Применяем формулу:
	\[
		\int_{-\infty}^{+\infty}R(x)dx = 2\pi i \sum\limits_{m}\underset{z_m}{\text{res}}
		R(z)
	\]
	Сумма вычетов берется по всем полюсам полуплоскости $Im{z}> 0$. Преобразуем
	исходный интеграл:
	\[
		\int_{-\infty}^{+\infty}\frac{dx}{(x^{2} - x + 1)^{2}}= \int_{-\infty}^{+\infty}
		\frac{dz}{(z^{2} - z + 1)^{2}}= \int_{-\infty}^{+\infty}\frac{dz}{(z - \frac{1}{2}+
		\frac{i\sqrt{3}}{2})^{2} (z - \frac{1}{2}- \frac{i\sqrt{3}}{2})^{2}}
	\]
	Особые точки:
	\[
		z = \frac{1}{2}+ \frac{i\sqrt{3}}{2}\quad (Im{z}> 0); \quad z = \frac{1}{2}-
		\frac{i\sqrt{3}}{2}\quad (Im{z}< 0)
	\]
	Точка $z = \frac{1}{2}+ \frac{i\sqrt{3}}{2}$ является полюсом второго порядка
	и вычет в ней равен:
	\[
		\underset{z = \frac{1}{2} + \frac{i\sqrt{3}}{2}}{\text{res}}f(z) = \lim\limits
		_{z\rightarrow\frac{1}{2} + \frac{i\sqrt{3}}{2}}\frac{d}{dz}\left[f(z)(z - \frac{1}{2}
		- \frac{i\sqrt{3}}{2})^{2} \right] = \lim\limits_{z\rightarrow\frac{1}{2} +
		\frac{i\sqrt{3}}{2}}\frac{d}{dz}\left[\frac{1}{(z - \frac{1}{2}+
		\frac{i\sqrt{3}}{2})^{2}}\right] =
	\]
	\[
		\lim\limits_{z\rightarrow\frac{1}{2} + \frac{i\sqrt{3}}{2}}\left[\frac{-16}{(2z
		- 1 + i\sqrt{3})^{3}}\right] = \frac{2\sqrt{3}}{9i}
	\]
	Используем приведунную в начале задачи формулу:
	\[
		\int_{-\infty}^{+\infty}\frac{dx}{(x^{2} - x + 1)^{2}}= 2\pi i \left(\frac{2\sqrt{3}}{9i}
		\right) = \frac{4\sqrt{3}\pi}{9}
	\]
	\subsection{Ответ: $\frac{4\sqrt{3}\pi}{9}$}

	\vspace{1cm}

	\section{Вычислить интеграл: }
	\[
		\int_{-\infty}^{+\infty}\frac{x^{3}\sin{x}}{x^{4} +5x^{2} + 4},\ dx
	\]
	\subsection{Решение:}
	Воспользуемся формулой
	\[
		\int_{-\infty}^{+\infty}R(x)\sin{\lambda}xdx = Im \left[2\pi i\sum\limits_{m}
		\underset{z_m}{\text{res}}R(z)e^{i\lambda z}\right], \lambda > 0
	\]
	Исходная функция полностью удовлетворяет условиям применения данной формулы. Необходимо
	найти $z_{m}$:
	\[
		x^{4} +5x^{2} + 4 = 0 \rightarrow z_{1,2}= \pm i; \quad z_{3,4}= \pm 2i;
	\]
	Сумма вычетов берется по верхней полуплоскости $Im{z}> 0$. Из этого следует $z_{m}
	= \{i; 2i\}$ Это особая точка является полюсом. Необходимо найти в ней вычет:

	\[
		1) \quad \underset{z = i}{\text{res}}R(z)e^{i\lambda z}= \lim\limits_{z
		\rightarrow i}\frac{z^{3}(z-1)}{(z^{2} + 1)(z^{2} + 4)}\cdot e^{iz}= \lim\limits
		_{z \rightarrow i}\frac{z^{3}e^{iz}}{(z + i)(z^{2} + 4)}= \frac{-ie^{-1}}{(i
		+ i)(-1 + 4)}= \frac{-i}{6i}e^{-1}= -\frac{1}{6}e^{-1}
	\]

	\[
		2) \quad \underset{z = 2i}{\text{res}}R(z)e^{i\lambda z}= \lim\limits_{z
		\rightarrow 2i}\frac{z^{3}(z-2i)}{(z^{2} + 1)(z^{2} + 4)}\cdot e^{iz}= \lim\limits
		_{z \rightarrow 2i}\frac{z^{3}e^{iz}}{(z + 2i)(z^{2} + 1)}= \frac{-8ie^{-2}}{(2i
		+ 2i)(-1 + 4)}= \frac{-8i}{-12i}e^{-2}= -\frac{2}{3}e^{-2}
	\]

	\[
		\int_{-\infty}^{+\infty}\frac{x^{3}\sin{x}}{x^{4} +5x^{2} + 4},\ dx = Im \left
		[2\pi i\sum\limits_{m}\underset{z_m}{\text{res}}R(z)e^{i\lambda z}\right] = \frac{4\pi}{3}
		e^{-2}+ \frac{\pi}{3}e^{-1}
	\]
	\subsection{Ответ: $\quad \frac{4\pi}{3}e^{-2}+ \frac{\pi}{3}e^{-1}$}

	\vspace{1cm}
	\section{Найти оригинал по заданному изображению: $\frac{1}{p^{3} - 1}$ }
	\subsection{Решение:}
	Необходимо представитть выражение, используя метод неопределенных
	коэффициентов:
	\[
		\frac{1}{p^{3} - 1}= \frac{1}{(p - 1)(p^{2} + p + 1)}= \frac{A}{p-1}+ \frac{Bp
		+ C}{p^{2} + p + 1}=
	\]
	\[
		= \frac{Ap^{2} + Ap + A + Bp^{2} - Bp + Cp - C}{(p-1)(p^{2} + p + 1)}
	\]
	\[
		= \frac{(A + B)p^{2} + (A-B+C)p + A - C}{(p-1)(p^{2} + p + 1)}
	\]

	\[
		\begin{cases}
			A + B = 0     \\
			A - B + C = 0 \\
			A - C = 1
		\end{cases}
		\longrightarrow
		\begin{cases}
			A = \frac{1}{3}   \\
			B = -\frac{1}{3}  \\
			C = - \frac{2}{3}
		\end{cases}
	\]
	Таким образом:
	\[
		\frac{1}{p^{3} - 1}= \frac{1}{3}\cdot \frac{1}{p-1}-\frac{1}{3}\cdot \frac{p}{p^{2}
		+ p + 1}- \frac{2}{3}\cdot \frac{1}{p^{2} + p + 1}
	\]

	\[
		\frac{1}{3}\cdot \frac{1}{p-1}-\frac{1}{3}\cdot \frac{p}{(p + \frac{1}{2})^{2}
		+ \frac{3}{4}}- \frac{2}{3}\cdot \frac{1}{(p + \frac{1}{2})^{2} + \frac{3}{4}}
	\]

	\[
		\frac{1}{3}\cdot \frac{1}{p-1}-\frac{1}{3}\cdot \frac{p + \frac{1}{2}}{(p + \frac{1}{2})^{2}
		+ \frac{3}{4}}- \frac{1}{\sqrt{3}}\cdot \frac{\frac{\sqrt{3}}{2}}{(p + \frac{1}{2})^{2}
		+ \frac{3}{4}}\rightarrow
	\]

	\[
		\rightarrow \frac{1}{3}e^{i2}- \frac{1}{3}e^{-i2}cos\frac{\sqrt{3}}{2}t - \frac{1}{\sqrt{3}}
		e^{-1}sin\frac{\sqrt{3}}{2}t
	\]

	\subsection{Ответ: $\quad 1\frac{1}{3}e^{i2}- \frac{1}{3}e^{-i2}cos\frac{\sqrt{3}}{2}
	t - \frac{1}{\sqrt{3}}e^{-1}sin\frac{\sqrt{3}}{2}t$}

	\vspace{1cm}
	\section{ Найти решения дифференциального уравнения, удовлетворяющее условиям:
	y'' - y = $\frac{1}{ch^{2}1}$ }

	\vspace{1cm}
	\section{Операционным методом решить задачу Коши
	$y'' + y' - 2y = e^{-t}, y(0) = -1, y'(0) = 1$ }
	\subsection{Решение:}

	\[
		p^{2}Y(p) - py(0) - y'(0) + pY(p) - y(0) -2Y(p) = \frac{1}{p+1}
	\]
	\[
		p^{2}Y(p) + p + pY(p)+1 - 2Y(p) = \frac{1}{p+1}
	\]
	\[
		(p^{2} + p - 2)Yp(p) = (p+2)(p-1)Y(p) = \frac{1}{p+1}-p-1 = \frac{-p(p+2)}{p+1}
	\]
	\[
		Y(p) = \frac{-p(p+2)}{(p+1)(p+2)(p-1)}= \frac{-P}{(p+1)(p-1)}= \frac{-P}{2}(\frac{1}{p-1}
		- \frac{1}{p+1})
	\]

	\text{ Найдем оригинал $y(t)$}

	\[
		Y(p) = -\frac{P}{2}\left(\frac{1}{p-1}- \frac{1}{p+1}\right) = \frac{1}{2}\frac{p}{p+1}
		-\frac{1}{2}\frac{p}{p-1}= \frac{1}{2}\left(1 - \frac{1}{p+1}\right) - \frac{1}{2}
		\left(1 + \frac{1}{p-1}\right) = -\frac{1}{2}\frac{1}{p+1}- \frac{1}{2}\frac{1}{p-1}
	\]
	\[
		\rightarrow y(t) = -\frac{1}{2}e^{-t}- \frac{1}{2}e^{t}
	\]

	\subsection{Ответ: $\quad y(t) = -\frac{1}{2}e^{-t}- \frac{1}{2}e^{t}$}

	\vspace{1cm}
	\section{Найти решение системы дифференциальных уравнений, удовлетворяющее заданному
	начальному условию }
	\[
		\begin{cases}x` = x + 2y \quad\quad x(0) = 0\\ y` = 2x + y + 1 \quad\quad y(0
		) = 5 \end{cases}
	\]
	\subsection{Решение:}
	\[
		\begin{cases}
			pX(p) - x(0) = X(p) + 2Y(p)               \\
			pY(p) - y(0) = 2X(p) + Y(p) + \frac{1}{p}
		\end{cases}
		\vspace{0.5cm}
	\]

	\text{поставим начальные условия}
	\vspace{0.3cm}
	\[
		\begin{cases}
			pX(p) = X(p) + 2Y(p)                   \\
			pY(p) - 5 = 2X(p) + Y(p) + \frac{1}{p}
		\end{cases}
		\vspace{0.4cm}
	\]

	\text{Выразим x(0) через y(p), использя второе уравнение}
	\[
		pY(p) - 5 = 2X(p) + Y(p) + \frac{1}{p}
	\]
	\[
		X(p) = \frac{pY(p) - Y(p) - 5 - \frac{1}{p}}{2}
	\]
	\[
		p \cdot\frac{pY(p) - Y(p) - 5 - \frac{1}{p}}{2}= \frac{pY(p)- Y(p) - 5 -
		\frac{1}{p}}{2}+ 2Y(p)
	\]
	\[
		Y(p) = \frac{5p-4-\frac{1}{p}}{p^{2} - 2p - 3}
	\]
	\[
		Y(p) = \frac{5p-4-\frac{1}{p}}{p^{2} - 2p - 3}= \frac{5p-4-\frac{1}{p}}{(p-1)^{2}
		- 4}- \frac{1}{3p}+ \frac{1}{3p}= \frac{1}{3}\frac{14p - 10}{(p-1)^{2} - 4}+
		\frac{1}{3p}= \frac{14}{3}\frac{p-1}{(p-1)^{2} - 4}+ \frac{2i}{3i(p-1)^{2} -
		4}+ \frac{1}{3p}
	\]

	\[
		y(t) = \frac{14}{3}e^{t} \cos{2it}- \frac{2i}{3}e^{t} \sin{2it}+ \frac{1}{3}=
		\frac{14}{3}e^{t}\ch{2t}+ \frac{2}{3}e^{t}\sh{2t}+ \frac{1}{3}
	\]
	\[
		y' = 2x + y + 1 \rightarrow x(t) = \frac{1}{2}(y' - y - 1) = \frac{1}{2}(6e^{t}
		\ch2t + 10e^{t}\sh{2t}- \frac{14}{3}e^{t}\ch{2t}- \frac{2}{3}e^{t}\sh{2t}-\frac{1}{3}
		- 1)
	\]
	\[
		= \frac{2}{3}e^{t}\ch{2t}+ \frac{14}{3}e^{\sh}{2t}- \frac{2}{3}
	\]

	\subsection{Ответ:
	\[
		x(t) = \frac{2}{3}e^{t}\ch{2t}+ \frac{14}{3}e^{t}\sh{2t}- \frac{2}{3}
	\]
	\[
		y(t) = \frac{14}{3}e^{t}\ch{2t}+ \frac{2}{3}e^{t}\sh{2t}+ \frac{1}{3}
	\]
	}
\end{document}
